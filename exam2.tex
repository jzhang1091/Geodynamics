\documentclass[12pt]{article}

% Any percent sign marks a comment to the end of the line

% Every latex document starts with a documentclass declaration like this
% The option dvips allows for graphics, 12pt is the font size, and article
%   is the style

\usepackage[pdftex]{graphicx}
\usepackage{indentfirst}
\usepackage{url}

% These are additional packages for "pdflatex", graphics, and to include
% hyperlinks inside a document.

\setlength{\oddsidemargin}{0.25in}
\setlength{\textwidth}{6.5in}
\setlength{\topmargin}{0in}
\setlength{\textheight}{8.5in}
\setlength{\parindent}{2em}

% These force using more of the margins that is the default style

\begin{document}

% Everything after this becomes content
% Replace the text between curly brackets with your own

\title{Project Proposal for Geodynamics}
\author{Jia Zhang}
\date{\today}

% You can leave out "date" and it will be added automatically for today
% You can change the "\today" date to any text you like


\maketitle

% This command causes the title to be created in the document

\section{Introduction}

% An article style is separated into sections and subsections with
%   markup such as this.  Use \section*{Principles} for unnumbered sections.

The mechanism of deep earthquakes remains one of the major questions in
seismology since their discovery from 1920s (Turner, 1922; Wadati, 1928).
It has long been recognized that the occurrence of deep earthquakes
is problematic, since brittle failure should be prohibited by confining pressure
at great depth. Rock strength does increase with pressure, but a few hundred MPa,
which is equivalent to 10-20 km depth, suffices to inhibit most fracture, and
elevated temperature activates ductile mechanisms that operate at stresses less
than the fracture strength
\cite{green1989}.

In addition, deep earthquakes show some differences with shallow brittle faulting earthquakes,
such as a dependence of source duration and rise time on earthquake depth (Houston and Williams, 1991),
much lower aftershock production rates (Page, 1968), and a dependence of magnitude-frequency
relations and aftershock productivity on slab temperature (Wiens and Gilbert, 1996)
\cite{wiens2001}.

Many different mechanisms have been proposed over the years for deep earthquakes, including plastic instabilities(Bridgman 1936), shear-induced melting(Griggs 1954), instabilities accompanying
recrystallization(Post 1977; Ogawa 1987) and polymorphic phase transformation(Bridgman 1954)
\cite{green1995}.
The phase change hypothesis was more highly populated with the assumption of a sudden implosion radiate the seismic energy without faulting. However, seismic evidence now precludes this possibility because deep earthquakes have double-couple motions similar to those of shallow events. Therefore, a new theory should be developed to explain the mechanism of the deep earthquakes.



\section{Targets}

Nowadays, seismological observations, in combination with mineral
physics experiments and geodynamic calculations provide some new views of
the mechanism and interpretation of deep earthquakes.

As for this project, I will review some basic seismological observations
concerning deep earthquakes(deeper than 300km), and discuss their
implications for the mechanism of deep earthquakes and the geodynamics of plate motion.

Here are some targets I am going to archive:
\begin{itemize}
\item Comparison of deep and shallow earthquakes
\item Previous theories of mechanisms and their limitation
\item New understanding of mechanism of deep earthquakes with new seismological observations
\end{itemize}




\begin{thebibliography}{99}

\bibitem{green1989} H. W. Green, P. C. Burnley,
{A new self-organizing mechanism for deep-focus earthquakes},
Nature, {\bf 343}, 733-737 (1989).

\bibitem{wiens2001} Douglas A. Wiens,
{Seismological constraints on the mechanism of deep earthquakes:
temperature dependence of deep earthquake source properties},
Phys. Earth Planet. Inter. {\bf 127}, 145-163 (2001).

\bibitem{green1995} H. W. Green, Heidi Houston,
{The mechanics of deep earthquakes},
Annu. Rev. Earth Planet. Sci. {\bf 23}, 169-213 (1995).


\end{thebibliography}



\end{document}
